\documentclass{article}
\usepackage[utf8]{inputenc}
\usepackage{amsmath}
\usepackage{amssymb}

\title{\textbf{ÁLGEBRA LINEAL}}

\date{Marzo 2022}
\begin{document}
\maketitle



\section{Vectores}
Formula
\[
    \displaystyle\sum\limits_{i=1}^n a_{i}b_{i}
\]
supongamos que tenemos dos vectores
tanto la suma como la resta de vectores esta definida por: \\


\subsection{Suma de Vectores}
la suma del primer índice de cada vector mas el primer índice del segundo vector,
lo mismo con el segundo índice de ambos vectores
lo mismo aplica para la resta el mismo orden en las operaciones
\[
    \vec{u} = ( -2,5) \: y \:
    \vec{v} =  (3,1)
\]
la suma esta dada por
la suma del primer índice de cada vector mas el primer índice del segundo vector,
lo mismo con el segundo índice de ambos vectores
\[
\vec{u} +\vec{v} = (-2 + 3, 5 - 1) = (1,4)
\]
\subsubsection{Un ejemplo mas de suma}
\[
\vec{A} = \begin{bmatrix}0 & 1 \\2 & 3 \end{bmatrix} \vec{B} = \begin{bmatrix}4 & 5 \\6 & -7 \end{bmatrix}
\]
\[
\vec{A} +\vec{B} =\begin{bmatrix}0 & 1 \\2 & 3 \end{bmatrix} + \begin{bmatrix}4 & 5 \\6 & -7 \end{bmatrix}
\]
\[
    =\begin{bmatrix}0+4 \\2+6 \end{bmatrix} + \begin{bmatrix}1+5 \\3+(-7) \end{bmatrix} = \begin{bmatrix}4 &  6 \\8 & -4 \end{bmatrix}
\]

\subsection{Resta de Vectores}
la resta esta dada por

\[
\vec{u} - \vec{v} = (-2 - 3, 5 - (-1)) = (-5,6)
\]

\subsection{Multiplicación por un escalar}
Multiplicación por un escalar
Supongamos que tenemos el escalar 5
con un vector (-2,3,4), la operación quedaría de la siguiente manera

\[
    5 (-2, 3, 4) = 5 (-10, 15, 20)
    -50,
\]

Donde multiplica el escalar por cada individuo del vector

\subsection{Producto Punto}
El producto punto o producto escalar de dos vectores es una operación que da como resultado un número real.
una de ellas es por medio de multiplicar el producto de los módulos de los vectores por el coseno del ángulo que forman, esto es
\[
    \vec{u} \cdot \vec{v} = | \vec{u}| | \vec{v}| cos \alpha
\]
Sin embargo, la forma más común de definir el producto punto no es esa,
 sino por medio de la suma de los productos de sus respectivas coordenadas,
  es decir,
  \[
  \vec{u} = (u_1, u_2, \dots, u_n) \: y \: \vec{v} = (v_1, v_2, \dots, v_n),
  \]
  entonces podemos definir el producto punto como
  \[
  \displaystyle \vec{u} \cdot \vec{v} = u_1v_1 + u_2v_2 + \cdots + u_nv_n = \sum_{i=1}^{n}{u_iv_i}
  \]

  Encuentra el producto punto de dos vectores, que tienen las coordenadas

  \[
    \displaystyle \vec{v} = \left( 1, \frac{1}{2}, 3\right), \qquad \vec{u} = \left(4, -4, 1\right).
    \]

    \[
         \vec{v} \cdot \vec{u} = \left( 1, \frac{1}{2}, 3\right) \cdot \left(4, -4, 1\right)\\ = (1)(4) + \left( \frac{1}{2}\right) (-4) + (3)(1)\\ = 4 - 2 + 3\\ = 5
    \]

\section{Sistema de ecuaciones lineales}


\subsection{Solución Única}

\subsubsection{1er Ejemplo}
Resolver el siguiente Sistema de ecuaciones lineales mediante
el método de Gauss-Jordán
\[
  2x + 3y + z = 1
  3x-2y-4z= -3
  5x-y-z = 4
\]

para solucionarla primero escribimos la matriz aumentada del Sistema

\[
    \begin{bmatrix}2 & 3 & 1 & 1 \\ 3  & -2 & -4 & -3 \\ 5  & -1 & -1  & 4 \end{bmatrix}
\]



Debemos llevar dicha matriz a su forma escalonada reducida mediante operaciones elementales.
para esto, escribiremos la matriz y a continuación una flecha. encima de esta flecha indicaremos las operaciones
que estamos efectuando
Notación:

\[
    cR_i \space nuevo renglo|n i de la matriz aumentada \space
    R_i <-> R_j intercambio del renglo|n i  con el renglo|n j
    aR_1 + R_j nuevo renglo|n j de la matriz aumentada
\]

desarrollo para obtener la forma escalonada reducida


\[
    \begin{bmatrix}2 & 3 & 1 & 1 \\ 3  & -2 & -4 & -3 \\ 5  & -1 & -1  & 4 \end{bmatrix}
    ->
    \begin{bmatrix}1 & \frac{3}{2} & \frac{1}{2} & \frac{1}{2} \\ 3  & -2 & -4 & -3 \\ 5  & -1 & -1  & 4 \end{bmatrix}
    -> -3R_1 + R_1
    -5R_1+R_3
    \begin{bmatrix} 1 & \frac{3}{2} & \frac{1}{2} & \frac{1}{2} \\ 0  &  - \frac{13}{2} & - \frac{11}{2} & -\frac{9}{2} \\ 0  & -\frac{17}{2} & -\frac{7}{2}  & \frac{3}{2} \end{bmatrix}
\]


\[
    -> -\frac{2}{13} \ space 2R_3
    \begin{bmatrix}1 & \frac{3}{2} & \frac{1}{2} & \frac{1}{2} \\ 0  & 1 & \frac{11}{13} & \frac{9}{13} \\ 0  & -17 & -7  & 3 \end{bmatrix}
\]

\[
    -> -17R_2 + R_3 \ space 2R_3
    \begin{bmatrix}1 & \frac{3}{2} & \frac{1}{2} & \frac{1}{2} \\ 0  & 1 & \frac{11}{13} & \frac{9}{13} \\ 0  & 0 & \frac{96}{13}  & \frac{192}{13} \end{bmatrix}
    -> \frac{13}{96}R_3
    \begin{bmatrix}1 & \frac{3}{2} & \frac{1}{2} & \frac{1}{2} \\ 0  & 1 & \frac{11}{13} & \frac{9}{13} \\ 0  & 0 & 1  & 2 \end{bmatrix}
\]

\[
    -> -\frac{11}{13} R_3 + R_2 \ space -\frac{1}{2}R_3+R1
    \begin{bmatrix}1 & \frac{3}{2} & \frac{1}{2} & -\frac{1}{2} \\ 0  & 1 & 0 & -1 \\ 0  & 0 & \frac{96}{13}  & \frac{192}{13} \end{bmatrix}
    -> -\frac{3}{1}R_2 +R_1
    \begin{bmatrix}1 & 0 & 0 & 1 \\ 0  & 1 & 0 & -1 \\ 0  & 0 & 1  & 2 \end{bmatrix}
    \]

\[
    x = 1, y = 1 z = 2
\]

Interpretación del resultado
la ultima matriz escalonada reducida indica que la solución del sistema es
x = 1, y = 1 z = 2


\subsubsection{2do Ejemplo}

\[
    3x +2y +4z = 1
    5x-y-3z = -7
    4x +3y + z = 2
\]

\[
    \begin{bmatrix}3 & 2 & 4 & 1 \\ 5  & -1 & -3 & -7 \\ 4  & 3 & 1  & 2 \end{bmatrix}
\]


El primer elemento del primer renglo|n que queremos que sea uno, una manera de obtenerlo es dividiendo entre 3,
sin embargo, no es el único camino ni el mejor para obtenerlo, en este caso obtendremos -1, primero y después haremos cero los demás elementos
de la primera columna posteriormente obtendremos 1


\[
    \begin{bmatrix}3 & 2 & 4 & 1 \\ 5  & -1 & -3 & -7 \\ 4  & 3 & 1  & 2 \end{bmatrix}
    -> -R_3+R_1
    \begin{bmatrix}-1 & -1  & 3 & -1 \\ 5  & -1 & -3 & -7 \\ 4  & 3 & 1  & 2 \end{bmatrix}
\]

\[
    -> 5R_1 + R_2 SPACE 4R_1 + R_3
    \begin{bmatrix} -1 & -1  & 3 & -1 \\ 0  & -6 & 12 & -12 \\ 0  & -1 & 13  & -2 \end{bmatrix}
    -> -R_1 space - \frac{1}{6} R_2
    \begin{bmatrix} 1 & 1  & -3 & 1 \\ 0  & 1 & -2 & 2 \\ 0  & -1 & 13  & -2 \end{bmatrix}
\]



\[
        -> R_2 + R_3
        \begin{bmatrix} 1 & 1  & -3 & 1 \\ 0  & 1 & -2 & 2 \\ 0  & 0 & 11  & 0 \end{bmatrix}
\]
\[
        -> \frac{1}{11} R_3
        \begin{bmatrix} 1 & 1  & -3 & 1 \\ 0  & 1 & -2 & 2 \\ 0  & 0 & 1  & 0 \end{bmatrix}
\]

\[
        -> 2R_3 + R_1 SPACE 3R_3 +R_1
        \begin{bmatrix} 1 & 1  & 0 & 1 \\ 0  & 1 & 0 & 2 \\ 0  & 0 & 1  & 0 \end{bmatrix}
\]

\[
        -> -R_1 + R_1
        \begin{bmatrix} 1 & 0  & 0 & -1 \\ 0  & 1 & 0 & 2 \\ 0  & 0 & 1  & 0 \end{bmatrix}
\]

La solución al sistema es 
x = 2, y = 2 z = 0

\subsubsection{3er Ejemplo}

Resuelva el siguiente sistema de ecuaciones lineales
\[
    a - b = 6
    b + c = 3
    c + 2d = 4
    2a - 3d = 5
\]

Escribiendo la matriz aumentada del sistema y reduciendo de acuerdo a la operación indicada tenemos

\[
    \begin{bmatrix}1 & -1 & 0 & 0 & -6  \\ 0 & 1 & 1 & 0 & 3 \\ 0 & 0 & 1 & 2 & 4  \\ 2 & 0 & 0 & -3 & -5 \end{bmatrix}
    -> -2R_1 + R_4
    \begin{bmatrix}1 & -1 & 0 & 0 & -6  \\ 0 & 1 & 1 & 0 & 3 \\ 0 & 0 & 1 & 2 & 4  \\ 0 & 2 & 0 & -3 & 17 \end{bmatrix}
    -> -2R_2 + R_4
\]

\[
    \begin{bmatrix} 1 & -1 & 0 & 0 & -6  \\ 0 & 1 & 1 & 0 & 3 \\ 0 & 0 & 1 & 2 & 4  \\ 0 & 0 & -2 & -3 & 11 \end{bmatrix}
    -> 2R_3 + R_4
    \begin{bmatrix}1 & -1 & 0 & 0 & -6  \\ 0 & 1 & 1 & 0 & 3 \\ 0 & 0 & 1 & 2 & 4  \\ 0 & 0 & 0 & 1 & 19 \end{bmatrix}
    -> -2R_4 + R_3
\]

\[
    \begin{bmatrix} 1 & -1 & 0 & 0 & -6  \\ 0 & 1 & 1 & 0 & 3 \\ 0 & 0 & 1 & 0 & -34  \\ 0 & 0 & 0 & 1 & 19 \end{bmatrix}
    -> -R_3 + R_1
    \begin{bmatrix} 1 & -1 & 0 & 0 & -6  \\ 0 & 1 & 0 & 0 & 37 \\ 0 & 0 & 1 & 0 & -34  \\ 0 & 0 & 0 & 1 & 19 \end{bmatrix}
    -> R_1 + R_1
\]
\[
    \begin{bmatrix} 1 & 0 & 0 & 0 & 31  \\ 0 & 1 & 0 & 0 & 37 \\ 0 & 0 & 1 & 0 & -34  \\ 0 & 0 & 0 & 1 & 19 \end{bmatrix}
\]

a = 31, b = 37, c = -34, d = 19

\subsection{Sin Solución}


\subsubsection{1er Ejemplo}

Resolver el siguiente sistema de ecuaciones

\[
    x+8y -5z = 3
    3x-2y+3z = 1
    2x +3y -z = 4
\]

Solución

\[
    \begin{bmatrix} 1 & 8 & -5 & 3   \\ 3 & -2 & 3 & 1  \\ 2 & 3 & -1 & 4  \end{bmatrix}
    -3R_1 + R_2 SPACXIOP -2R_1 + R_3
    \begin{bmatrix} 1 & 8 & -5 & 3   \\ 0 & -26 & 18 & -8  \\ 0 & -13 & 9 & -2  \end{bmatrix}
\]

\[
    -2R_3 + R_2
    \begin{bmatrix} 1 & 8 & -5 & 3   \\ 0 & 0 & 0 & -4  \\ 0 & -13 & 9 & -2  \end{bmatrix}
\]

No hay necesidad de seguir reduciendo, del segundo renglo|n se tiene 0x + 0y + 0z = -4
que da la igualdad 0 = -4 CONTRADICCION, por lo tanto, el sistema no tiene Solución

\subsubsection{2do Ejemplo}

Resolver el siguiente sistema de ecuaciones

\[
    \begin{bmatrix}

         a & b & -3c & -4d = -1  \\
         2a & 2b & -c & -2d = 1  \\
         a & b & 2c &   2d  =5 \\

    \end{bmatrix}
\]

solución

\[
    \begin{bmatrix}

         1 & 1 & -3 & -4 = -1  \\
         2& 2 & -1 & -2 = 1  \\
         1 & 1 & 2 &   2  =5 \\

    \end{bmatrix}
    -> -2R_1 +R_2 spacio -R_1+R_3
    \]
    \[
        \begin{bmatrix}
            1 & 1 & -3 & -4 = -1  \\
            0& 0 & 5 & 6 = 3  \\
            0 & 0 & 5 &  6  =6 \\
        \end{bmatrix}
        \]

        \[
    -> -R_2 +R_3
    \begin{bmatrix}
         1 & 1 & -3 & -4 = -1  \\
         0& 0 & 5 & 6 = 3  \\
         0 & 0 & 0 &  0  =3 \\
    \end{bmatrix}
\]

Del tercer renglo|n se tiene 0a + 0b + 0c + 0d = 3
el  sistema no tiene solución


\subsection{3er Ejemplo}
\[
\begin{bmatrix}
 x & +y & +z & -w = 2  \\
 2x& -y &  & +w = 5 \\
 3x &  & +z &+ w    =1 \\
 2x & +2y & +2z &  -w  =3 \\
\end{bmatrix}
\]

\[
\begin{bmatrix}
 1 & 1 & 1 & -1 &  2 \\
 2& -1 & 0 & 1  & 5 \\
 3 & 0 & 1 & 1  &   1 \\
 2 & 2 & 2 &  -1&   3 \\
\end{bmatrix}
 -> -2R_1 + R_2 SPACE -3R_1 + R_3 SPACE -2R_1 + R_4
\]


\[
\begin{bmatrix}
 1 & 1 & 1 & -1  & 2  \\
 0& -3 & -2 & 3  & 1 \\
 0 & -3 & -2 & 4   &  -5 \\
 0 & 0 & 0 &  1 &  -1 \\
\end{bmatrix}
-> -R_2+R_3
\]


\[
\begin{bmatrix}
 1 & 1 & 1 & -1  & 2  \\
 0& -3 & -2 & 3  & 1 \\
 0 & 0 & 0 & 1   &  -6 \\
 0 & 0 & 0 &  1 &  -1 \\
\end{bmatrix}
-> -R_3+R_4
\]
\[
\begin{bmatrix}
 1 & 1 & 1 & -1  & 2  \\
 0& -3 & -2 & 3  & 1 \\
 0 & 0 & 0 & 1   &  -6 \\
 0 & 0 & 0 &  0 &  5 \\
\end{bmatrix}
\]

del cuarto renglo|n se tiene 0x + 0y + 0z + 0w = 5
que da la igualdad 0 = 5
por lo tanto, el sistema no tiene solución



\subsection{Infinititas Soluciones}

\subsubsection{1er Ejemplo}

obtener la solución del siguiente Sistema de ecuaciones lineales

\[
\begin{bmatrix}
 3x & -2y & + 3z = 5  \\
 2x & 4y & -z = 2  \\
\end{bmatrix}
\]

Solución

\[
\begin{bmatrix}
 3 & -2 & + 3 & 5  \\
 2 & 4 & -1 & 2  \\
\end{bmatrix}
-> -R_2+ R_1
\]
\[
\begin{bmatrix}
 1 & -6 & + 4 & 3  \\
 2 & 4 & -1 & 2  \\
\end{bmatrix}
-> -R_1+ R_2
\]
\[
\begin{bmatrix}
 1 & -6 & + 4 & 3  \\
 0 & 16 & -9 & -4  \\
\end{bmatrix}
-> \frac{1}{16}R_2
\]

\[
\begin{bmatrix}
 1 & -6 & + 4 & 3  \\
 0 & 1 & \frac{9}{16} & -\frac{1}{4}  \\
\end{bmatrix}
-> 6R_2+R_1
\]
\[
\begin{bmatrix}
 1 & 0 & + \frac{5}{8} & \frac{3}{2}  \\
 0 & 1 & \frac{9}{16} & -\frac{1}{4}  \\
\end{bmatrix}
\]
La última matriz está en su forma escalonada reducida, ya no se puede reducir más,
de donde obtendremos

\[
    x + \frac{5}{8}z = \frac{3}{2}
    y - \frac{9}{16}z = -\frac{1}{4}
\]
Despejando x,y
\[
    x  =  \frac{3}{2} -\frac{5}{8}z
    y = -\frac{1}{4} + \frac{9}{16}z
\]
luego x,y dependen de z, si z = t, tenemos Que t pertenece a los numero reales R

\[
    x  =  \frac{3}{2} -\frac{5}{8}t
    y = -\frac{1}{4} + \frac{9}{16}t
    z = t
\]
es decir, el sistema de ecuaciones tiene una infinidad de soluciones ya que para cada valor
de t habrá un valor para x,y,z


por ejemplo, si t = 0 entonces:
\[
X = \frac{3}{2}
y = -\frac{1}{4}
z = 0
\]
es una solución para el sistema de ecuaciones
sí t = 1 entonces
\[
X = \frac{7}{8}
y = \frac{5}{16}
z = 1
\]

\subsubsection{3er Ejemplo}


\[
\begin{bmatrix}
 2x & -y & + 3z = 4  \\
 3x & +2y & -z = 3  \\
 X & +3y & -4z = -1  \\
\end{bmatrix}
\]

solución
Desde un principio podemos escribir la matriz aumentada del sistema con su primer
renglo|n de coeficientes de la tercera ecuacion, es decir, podemos reordenar las ecuaciones

\[
\begin{bmatrix}
 1 & 3 & -4 & -4  \\
 2 & -1 & 3 & 4  \\
 3 & 2 & -1 & 3  \\
\end{bmatrix}
-> -2R_1+ R_2 space  -3R_1+R_3
\]

\[
\begin{bmatrix}
 1 & 3 & -4 & -1  \\
 0 & -7 &11 & 6  \\
 0 & -7 & 11 & 6  \\
\end{bmatrix}
-> -R_2+ R_3
\]

\[
\begin{bmatrix}
 1 & 3 & -4 & -1  \\
 0 & -7 &11 & 6  \\
 0 & 0 & 0 & 0  \\
\end{bmatrix}
-> -\frac{1}{7}R_2
\]

\[
\begin{bmatrix}
 1 & 3 & -4 & -1  \\
 0 & 1 & -\frac{11}{7} & -\frac{6}{7}  \\
 0 & 0 & 0 & 0  \\
\end{bmatrix}
\]


\[
    x + \frac{5}{7}z = \frac{11}{7}
    y - \frac{11}{7}z = -\frac{6}{7}
    0z = 0
\]

\[
    x + \frac{5}{7}z = \frac{11}{7}
    y - \frac{11}{7}z = -\frac{6}{7}
\]

\[
    x = \frac{11}{7} - \frac{5}{7}z
    y = -\frac{6}{7} +\frac{11}{7}z 
\]
Nuevamente pongamos z = t, donde t pertenece a los reales entonces el conjunto queda expresado como
\[
    x = \frac{11}{7} - \frac{5}{7}t
    y = -\frac{6}{7} +\frac{11}{7}t
    z = t
\]

por lo tanto, el sistema tiene una infinidad de soluciones



\subsection{Ejercicios de SEL por el método de la inversa con GAUSS--JORDAN }


\subsubsection{1er Ejemplo}

\[
A =
\begin{bmatrix}
 1 & 2  \\
 1 & 3  \\
\end{bmatrix}
\]

lo primero que debemos de hacer es poner la matriz A y la matriz identidad en una 
sola matriz. La matriz A en la parte izquierda y la matriz identidad en la parte derecha

\[
A =
\begin{bmatrix}
 1 & 2  \\
 1 & 3  \\
\end{bmatrix}
I =
\begin{bmatrix}
 1 & 0  \\
 0 & 1  \\
\end{bmatrix}
\]
Ahora para calcular la matriz inversa, tenemos que convertir la matriz de la parte izquierda
en la matriz identidad, y para ello, debemos aplicar transformaciones en las filas hasta conseguirlo

En el primer término de todos, el 1 ya que es igual a la matriz identidad. por lo tanto, no hace falta aplicar 
ninguna transformación en la primera fila de momento

la matriz identidad tiene un 0 en el último elemento de la primera columna, desde ahora tenemos un 1
así que tenemos que convertir el 1 en un 0 para ello a la fila 2 le restamos 1
\[
A =
\begin{bmatrix}
 1 & 2  \\
 1 & 3  \\
\end{bmatrix}
I =
\begin{bmatrix}
 1 & 0  \\
 0 & 1  \\
\end{bmatrix}
-> f2-f1
\]

\[
A =
\begin{bmatrix}
 1 & 2  \\
 0 & 1  \\
\end{bmatrix}
I =
\begin{bmatrix}
 1 & 0  \\
 -1 & 1  \\
\end{bmatrix}
\]

Pasamos a la segunda columna el 1 de abajo ya está bien.
pero el 2 de arriba no, ya que la matriz identidad tiene un 0 en esa posición,
por lo tanto, para convertir el 2 en un 0, a la fila 1 le restamos la fila 2 multiplicada por 2

\[
\begin{bmatrix}
 1 & 0  \\
 0 & 1  \\
\end{bmatrix}
-> f1-2f2
\begin{bmatrix}
 3 & -2  \\
 -1 & 1  \\
\end{bmatrix}
\]

la matriz inversa es la matriz que obtenemos en la parte derecha tras convertir la matriz
de la izquierda en la matriz identidad. y ahora en la parte de la izquierda. por lo tanto la matriz inversa es



\[
A-1 =
\begin{bmatrix}
 3 & -2  \\
 -1 & 1  \\
\end{bmatrix}
\]


\subsubsection{2do Ejemplo}

\[
A =
\begin{bmatrix}
 1 & 2 & -4 \\
 0 & 3 & 2 \\
 0 & 1 & 1 \\
\end{bmatrix}
\]

\[
    A =
    \begin{bmatrix}
     1 & 2 & -4 \\
     0 & 3 & 2 \\
     0 & 1 & 1 \\
    \end{bmatrix}
I =
\begin{bmatrix}
 1 & 0 & 0 \\
 0 & 1 & 0 \\
 0 & 0 & 1 \\
\end{bmatrix}
-> f2-2f3
\]

\[
    A =
    \begin{bmatrix}
     1 & 1 & 4 &|& 1 & 0 & 0 \\
     0 & 1 & 0 &|& 0 & 1 & 0 \\
     0 & 1 & 1 &|& 0 & 0 & 1 \\
    \end{bmatrix}
    -> f2-2f3
\]
\[
    A =
    \begin{bmatrix}
     1 & 1 & 4 &| &1 & 0 & 0 \\
     0 & 1 & 0 &| &0 & 1 & -2 \\
     0 & 1 & 1 &| &0 & 0 & 1 \\
    \end{bmatrix}
    -> f3- f2
\]
\[
    A =
    \begin{bmatrix}
     1 & 1 & -4& |& 1 & 0 & 0 \\
     0 & 1 & 0 &| &0 & 1 & -2 \\
     0 & 0 & 1 &| &0 & -1 & 3 \\
    \end{bmatrix}
    -> f1- f2
\]

\[
    A =
    \begin{bmatrix}
     1 & 1 & -4 & |& 1 & -1 & 2 \\
     0 & 1 & 0  &| &0 & 1 & 2 \\
     0 & 0 & 1  &| &0 & -1 & 3 \\
    \end{bmatrix}
    -> f1+ 4f3
\]
\[
    A =
    \begin{bmatrix}
     1 & 0 & 0 |& 1 & -5 & 14 \\
     0 & 1 & 0 |& 0 & 1 & -2 \\
     0 & 0 & 1 |& 0 & -1 & 3 \\
    \end{bmatrix}
    -> f1+ 4f3
\]
\[
    A-1 =
    \begin{bmatrix}
      1 & -5 & 14 \\
      0 & 1 & -2 \\
      0 & -1 & 3 \\
    \end{bmatrix}
\]



\subsubsection{3ero Ejemplo}

\[
    A =
    \begin{bmatrix}
        1 & 2 & 1 \\
        0 & 1 & 0 \\
        2 & 0 & 3 \\
    \end{bmatrix}
\]

\[
    A =
    \begin{bmatrix}
        1 & 2 & 1 & | & 1 & 1 & 1 \\
        0 & 1 & 0 & | & 1 & 1 & 1 \\
        2 & 0 & 3 & | & 1 & 1 & 1 \\
    \end{bmatrix}
    -> f3-2f1
\]

\[
    A =
    \begin{bmatrix}
        1 & 2 & 1 & | & 1 & 0 & 0 \\
        0 & 1 & 0 & | & 0 & 1 & 0 \\
        2 & 4 & 1 & | & -2 & 0 & 1 \\
    \end{bmatrix}
    -> f1-2f2
\]

\[
    A =
    \begin{bmatrix}
        1 & 0 & 1 & | & 1 & -2 & 0 \\
        0 & 1 & 0 & | & 0 & 1 & 0 \\
        2 & -4 & 1 & | & -2 & 0 & 1 \\
    \end{bmatrix}
    -> f3+4f2
\]
\[
    A =
    \begin{bmatrix}
        1 & 0 & 1 & | & 1 & -2 & 0 \\
        0 & 1 & 0 & | & 0 & 1 & 0 \\
        0 & 0 & 1 & | & -2 & 4 & 1 \\
    \end{bmatrix}
    -> f1-f3
\]

\[
    A =
    \begin{bmatrix}
        1 & 0 & 0 & | & 3 & -6 & -1 \\
        0 & 1 & 0 & | & 0 & 1 & 0 \\
        0 & 0 & 1 & | & -2 & 4 & 1 \\
    \end{bmatrix}
\]

\[
    A-1 =
    \begin{bmatrix}
        3 & -6 & -1 \\
        0 & 1 & 0 \\
        -2 & 4 & 1 \\
    \end{bmatrix}
\]


\subsubsection{4to Ejemplo}


\[
    A =
    \begin{bmatrix}
        1 & -2 & 0 \\
        1 & 2 & 2 \\
        0 & 3 & 2 \\
    \end{bmatrix}
\]

\[
    A =
    \begin{bmatrix}
        1 & -2 & 0 & | & 1 & 0 & 0 \\
        1 & 2 & 2 & | & 0 & 1 & 0 \\
        0 & 3 & 2 & | & 0 & 0 & 1 \\
    \end{bmatrix}
    -> f2-f1
\]


\[
    A =
    \begin{bmatrix}
        1 & -2 & 0 & | & 1 & 0 & 0 \\
        0 & 4 & 2 & | & -1 & 1 & 0 \\
        0 & 3 & 2 & | & 0 & 0 & 1 \\
    \end{bmatrix}
    -> f2-f4
\]

\[
    A =
    \begin{bmatrix}
        1 & -2 & 0 & | & 1 & 0 & 0 \\
        0 & 1 & \frac{2}{4} & | & -\frac{1}{4} & \frac{1}{4} & 0 \\
        0 & 3 & 2 & | & 0 & 0 & 1 \\
    \end{bmatrix}
\]
a la fila 1 le sumamos la fila 2 multiplicada por 2
\[
    A =
    \begin{bmatrix}
        1 & -2 & 0 & | & \frac{2}{4} & \frac{2}{4} & 0 \\
        0 & 1 & \frac{2}{4} & | & -\frac{1}{4} & \frac{1}{4} & 0 \\
        0 & 3 & 2 & | & 0 & 0 & 1 \\
    \end{bmatrix}
\]
A la fila 3 le restamos la fila 2 multiplicada por 3
\[
    A =
    \begin{bmatrix}
        1 & 0 & 1 & | & \frac{2}{4} & \frac{2}{4} & 0 \\
        0 & 1 & \frac{2}{4} & | & -\frac{1}{4} & \frac{1}{4} & 0 \\
        0 & 0 & \frac{2}{4} & | & -\frac{3}{4} & \frac{3}{4} & 1 \\
    \end{bmatrix}
\]
multiplicamos la tercera fila por dos
\[
    A =
    \begin{bmatrix}
        1 & 0 & 1 & | & \frac{2}{4} & \frac{2}{4} & 0 \\
        0 & 1 & \frac{2}{4} & | & -\frac{1}{4} & \frac{1}{4} & 0 \\
        0 & 0 & 1 & | & \frac{6}{4} & -\frac{6}{4} & 2 \\
    \end{bmatrix}
\]

a la fila 2 le restamos la fila 3 dividida entre 2
\[
    A =
    \begin{bmatrix}
        1 & 0 & 1 & | & \frac{2}{4} & \frac{2}{4} & 0 \\
        0 & 1 & 0 & | & -1 & 1 & 0 \\
        0 & 0 & 1 & | & \frac{6}{4} & -\frac{6}{4} & 2 \\
    \end{bmatrix}
\]

A la fila 1 le restamos la fila 3
\[
    A =
    \begin{bmatrix}
        1 & 0 & 0 & | & -1 & 2 & -2 \\
        0 & 1 & 0 & | & -1 & 1 & -1 \\
        0 & 0 & 1 & | & \frac{6}{4} & -\frac{6}{4} & 2 \\
    \end{bmatrix}
\]
\[
    A-1 =
    \begin{bmatrix}
        -1 & 2 & -2 \\
        -1 & 1 & -1 \\
        \frac{6}{4} & -\frac{6}{4} & 2 \\
    \end{bmatrix}
\]

\subsubsection{5to Ejemplo}
Dada la matriz de dimensión 2x2
\[
    A =
    \begin{bmatrix}
        2 & 0  \\
        4 & -1 \\
    \end{bmatrix}
\]

la matriz por bloque es
\[
    A =
    \begin{bmatrix}
        2 & 0 &  | & 1 & 0 \\
        4 & -1 & | & 0& 1 \\
    \end{bmatrix}
\]
Dividimos la fila uno entre dos

\[
    A =
    \begin{bmatrix}
        1 & 0 &  | & \frac{1}{2} & 0 \\
        4 & -1 & | & 0& 1 \\
    \end{bmatrix}
\]

a la fila 2 le restamos el cuádruple de la fila 1

\[
    A =
    \begin{bmatrix}
        1 & 0 &  | & \frac{1}{2} & 0 \\
        0 & -1 & | & -2& 1 \\
    \end{bmatrix}
\]

multiplicamos la fila 2 por -1

\[
    A =
    \begin{bmatrix}
        1 & 0 &  | & \frac{1}{2} & 0 \\
        0 & -1 & | & 2& 1 \\
    \end{bmatrix}
\]

como tenemos la identidad del lado izquierda, la inversa de A es la matriz del lado derecho

\[
    A-1 =
    \begin{bmatrix}
       \frac{1}{2}  & 0  \\
        2 & -1 &  \\
    \end{bmatrix}
\]





\subsection{Factorizacion LU}




\subsubsection{1er Ejemplo}
Encontrar una factorizacionn de la forma PA = LU para la matriz


\[
    A=
    \begin{bmatrix}
        0 & 0  & 2   \\
        -1 & 5  & -2   \\
        3 & 6  & 7   \\
    \end{bmatrix}
\]

utilizando el pivoteo parcial.

\[
    \begin{bmatrix}
        0 & 0  & 2   \\
        -1 & 5  & -2   \\
        3 & 6  & 7   \\
    \end{bmatrix}
\]

-> Intercambiamos el primer renglon por el tercero

\[
\begin{bmatrix}
        3 & 6 & 7   \\
        -1 & 5  & -2   \\
        0 & 0  & 2   \\
    \end{bmatrix}
\]

-> sumandole un tercio a la segunda columna
\[
\begin{bmatrix}
        3 & 6 & 7   \\
        0 & 7  & \frac{1}{3}   \\
        0 & 0  & 2   \\
    \end{bmatrix}
\]

entonces

\[
    U=
    \begin{bmatrix}
        3 & 6  & 7   \\
       0 & 7  & \frac{1}{3}   \\
        0 & 0  & 2   \\
    \end{bmatrix}
\]

\[
    L=
    \begin{bmatrix}
        1 & 0  & 0   \\
        -\frac{1}{3} & 1  & 0   \\
        0 & 0  & 1   \\
    \end{bmatrix}
\]


para la martiz P, la matriz permutacion. observamos que cambiamos la fila 
1 y 3 es decir

\[
    P=
    \begin{bmatrix}
        1 & 0  & 0   \\
        0 & 1  & 0   \\
        0 & 0  & 1   \\
    \end{bmatrix}
\]


\[
    P=
    \begin{bmatrix}
        0 & 0  & 1   \\
        0 & 1  & 0   \\
        1 & 0  & 0   \\
    \end{bmatrix}
\]

se comprueba que 
\[
    \begin{bmatrix}
        0 & 0  & 1   \\
        0 & 1  & 0   \\
        1 & 0  & 0   \\
    \end{bmatrix}
    \begin{bmatrix}
        0 & 0  & 2   \\
        -1 & 5  & -2   \\
        3 & 6  & 7   \\
    \end{bmatrix}
\]

=

\[
    \begin{bmatrix}
        1 & 0  & 0   \\
       -\frac{1}{3}  & 1  & 0   \\
        0 & 0  & 1   \\
    \end{bmatrix}
    \begin{bmatrix}
        3 & 6 & 7   \\
        0 & 7  & \frac{1}{3}   \\
        0 & 0  & 2   \\
    \end{bmatrix}
\]

\subsubsection{2do Ejemplo}

Dos permutaciones de filas 
Encontrar una factorizacion de la forma PA = LU para la matriz

\[
    A=
    \begin{bmatrix}
        0 & 1  & 1   \\
        -1 & 2  & -4   \\
        2 & -5  & 1   \\
    \end{bmatrix}
\]

utilizando el pivoteo parcial
Cambiar el renglon 1 por el 3

\[
    A=
    \begin{bmatrix}
        2 & -5  & 1   \\
        -1 & 2  & -4   \\
        0 & 1  & 1   \\
    \end{bmatrix}
\]

multriplicarlo el renglon 2 por 1/2
\[
    A=
    \begin{bmatrix}
        2 & -5  & 1   \\
        0 & -\frac{1}{2}  & -\frac{7}{2}   \\
        0 & 1  & 1   \\
    \end{bmatrix}
\]
cambiar el renglon dos por el tercero
\[
    A=
    \begin{bmatrix}
        2 & -5  & 1   \\
        0 & 1  & 1   \\
        0 & -\frac{1}{2}  & -\frac{7}{2}   \\
    \end{bmatrix}
\]

multriplicar el 3er renglon por 1/2
\[
    A=
    \begin{bmatrix}
        2 & -5  & 1   \\
        0 & 1  & 1   \\
        0 & 0 & -3   \\
    \end{bmatrix}
\]

asi obtenemos
\[
    U=
    \begin{bmatrix}
        2 & -5  & 1   \\
        0 & 1  & 1   \\
        0 & 0 & -3   \\
    \end{bmatrix}
\]
la matriz L antes del segundo intercambio era
\[
    \begin{bmatrix}
        1 & 0  & 0   \\
        -\frac{1}{2} & 1  & 0   \\
        0 & 0 & 1  \\
    \end{bmatrix}
\]
pero al intercambiar las filas 2 y 3 la matriz L cambia a 
\[
    \begin{bmatrix}
        1 & 0  & 0   \\
        0 & 0 & 1  \\
        -\frac{1}{2} & 1  & 0   \\
    \end{bmatrix}
\]
despues de ese intercambio queda por agregar la ultima operacion entre filas y L
\[
  L =
\begin{bmatrix}
        1 & 0  & 0   \\
        0 & 1 & 0  \\
        -\frac{1}{2} & -\frac{1}{2}  & 1   \\
    \end{bmatrix}
\]
para la matriz P observamos que cambiamos primero la fila 1 y 3
luego la fila 2 y 3
\[
  P =
\begin{bmatrix}
        1 & 0  & 0   \\
        0 & 1 & 0  \\
        0 & 0  & 1   \\
    \end{bmatrix}
\]
\[
  P =
\begin{bmatrix}
    0 & 0  & 1   \\
    0 & 1 & 0  \\
        1 & 0  & 0   \\
    \end{bmatrix}
\]
\[
  P =
\begin{bmatrix}
    0 & 0  & 1   \\
    1 & 0  & 0   \\
    0 & 1 & 0  \\
    \end{bmatrix}
\]

se comprueba que
\[
\begin{bmatrix}
    0 & 0  & 1   \\
    1 & 0  & 0   \\
    0 & 1 & 0  \\
\end{bmatrix}
\begin{bmatrix}
    0 & 1  & 1   \\
    -1 & 2  & -4   \\
    2 & -5 & 1  \\
\end{bmatrix}
=
\begin{bmatrix}
    1 & 0  & 0   \\
    0 & 1 & 0  \\
    -\frac{1}{2} & -\frac{1}{2} & 1  \\
\end{bmatrix}
\begin{bmatrix}
    2 & -5  & 1   \\
    0 & 1  & 1   \\
    0 & 0 & -1  \\
\end{bmatrix}
\]






\subsubsection{3er Ejemplo}

Encontrar una factorizacion de la forma PA = LU para la matriz

\[
    A=
    \begin{bmatrix}
        2 & -2  & 1   \\
        -8 & 11  & 5   \\
        4 & 13 & 3   \\
    \end{bmatrix}
\]
utilizando el pivoteo parcial
\[
    A=
    \begin{bmatrix}
        2 & -2  & 1   \\
        -8 & 11  & 5   \\
        4 & 13 & 3   \\
    \end{bmatrix}
\]
\[
    A=
    \begin{bmatrix}
        -8 & 11  & 5   \\
        2 & -2  & 1   \\
        4 & 13 & 3   \\
    \end{bmatrix}
\]
sumando el segundo renglon por 1/4 y el 3ero por 1/2
\[
    A=
    \begin{bmatrix}
        -8 & 11  & 5   \\
        0 & \frac{3}{4}  & \frac{9}{4}   \\
        4 & -\frac{15}{2} & \frac{11}{2}   \\
    \end{bmatrix}
\]

\[
    A=
    \begin{bmatrix}
        -8 & 11  & 5   \\
        4 & -\frac{15}{2} & \frac{11}{2}   \\
        0 & \frac{3}{4}  & \frac{9}{4}   \\
    \end{bmatrix}
\]

    sumandole al 3er rengon 1/10
\[
    A=
    \begin{bmatrix}
        -8 & 11  & 5   \\
        4 & -\frac{15}{2} & \frac{11}{2}   \\
        0 & 0  & \frac{14}{5}   \\
    \end{bmatrix}
\]
luego
\[
    U=
    \begin{bmatrix}
        -8 & 11  & 5   \\
        4 & -\frac{15}{2} & \frac{11}{2}   \\
        0 & 0  & \frac{14}{5}   \\
    \end{bmatrix}
\]

la matriz L antes del segundo cambio de filas era

\[
    \begin{bmatrix}
        1 & 0  & 0   \\
        -\frac{1}{4} & 1 & 0   \\
        -\frac{1}{2} & 0  & 1   \\
    \end{bmatrix}
\]

pero al intercambiar las filas 2 y 3 la matriz cambia a

\[
    \begin{bmatrix}
        1 & 0  & 0   \\
        -\frac{1}{2} & 0  & 1   \\
        -\frac{1}{4} & 1 & 0   \\
    \end{bmatrix}
\]

despues de este intercambio queda por agregar la uiltima operacion entre fila
y L es 
\[
    L =
    \begin{bmatrix}
        1 & 0  & 0   \\
        -\frac{1}{2} & 1 & 0   \\
        -\frac{1}{4} & -\frac{1}{10}  & 1   \\
    \end{bmatrix}
\]

Para P la matriz de permutacion, observamos que cambiamos primero la fila
1 y 2 luego la fila 2 y 3 es decir

\[
    p =
    \begin{bmatrix}
        1 & 0  & 0   \\
     0 & 1 & 0   \\
       0 & 0  & 1   \\
    \end{bmatrix}
\]
\[
    p =
    \begin{bmatrix}
        0 & 1 & 0   \\
        1 & 0  & 0   \\
       0 & 0  & 1   \\
    \end{bmatrix}
\]
\[
    p =
    \begin{bmatrix}
        0 & 1 & 0   \\
        0 & 0  & 1   \\
        1 & 0  & 0   \\
    \end{bmatrix}
\]

Se comprueba que 
\[
    \begin{bmatrix}
        0 & 1 & 0   \\
        0 & 0  & 1   \\
        1 & 0  & 0   \\
    \end{bmatrix}
    \begin{bmatrix}
        2 & -2 & 1   \\
    -8 & 11  & 5   \\
        4 & -13  & 3   \\
    \end{bmatrix}
    =
    \begin{bmatrix}
        1 & 0 & 0   \\
    -\frac{1}{2}& 1  & 0   \\
       -\frac{1}{4}  & -\frac{1}{10}  & 1   \\
    \end{bmatrix}
    \begin{bmatrix}
        -8 & 11 & 5   \\
    0 & -\frac{15}{2}  & \frac{11}{2}   \\
       0  & 0  & \frac{14}{5}   \\
    \end{bmatrix}
\]



\subsubsection{4to Ejemplo}
tres permutaciones de filas 
\[
    A=
    \begin{bmatrix}
        1 & 4 & 0 & -4   \\
        5 & 1 & 1 & -1   \\
        3 & 1 & -1 & -2   \\
        -3 & 4 & 6 & 2   \\
    \end{bmatrix}
\]

utilizando El pivoteo parcial
cambiando la fila 1 por la fila 2
\[
    A=
    \begin{bmatrix}
        5 & 1 & 1 & -1   \\
        1 & 4 & 0 & -4   \\
        3 & 1 & -1 & -2   \\
        -3 & 4 & 6 & 2   \\
    \end{bmatrix}
\]

sumando -1/5 a la fila 2
sumandole -3/5 a la fila 3
sumando 3/5 a la cuarta fila
\[
    \begin{bmatrix}
        5 & 1 & 1 & -1   \\
        0 & \frac{19}{5}  & -\frac{1}{5} & -\frac{19}{5}   \\
        0 & \frac{2}{5} & -\frac{8}{5} & -\frac{7}{5}   \\
        0 & \frac{23}{5} & \frac{33}{5} & \frac{7}{5}   \\
    \end{bmatrix}
\]

\[
    \begin{bmatrix}
        5 & 1 & 1 & -1   \\
        0 & \frac{23}{5} & \frac{33}{5} & \frac{7}{5}   \\
        0 & \frac{2}{5} & -\frac{8}{5} & -\frac{7}{5}   \\
        0 & \frac{19}{5}  & -\frac{1}{5} & -\frac{19}{5}   \\
    \end{bmatrix}
\]

restanto -2/23 a la tercer fila
restando -19/23 a la 4ta fila
\[
    \begin{bmatrix}
        5 & 1 & 1 & -1   \\
        0 & \frac{23}{5} & \frac{33}{5} & \frac{7}{5}   \\
        0 & 0 & -\frac{50}{23} & -\frac{35}{23}   \\
        0 & 0  & -\frac{130}{23} & -\frac{114}{23}   \\
    \end{bmatrix}
\]
\[
    \begin{bmatrix}
        5 & 1 & 1 & -1   \\
        0 & \frac{23}{5} & \frac{33}{5} & \frac{7}{5}   \\
        0 & 0 & -\frac{50}{23} & -\frac{35}{23}   \\
        0 & 0  & -\frac{130}{23} & -\frac{114}{23}   \\
    \end{bmatrix}
\]

cambiando la fila 3 y 4 y despues restando -5/13
\[
    \begin{bmatrix}
        5 & 1 & 1 & -1   \\
        0 & \frac{23}{5} & \frac{33}{5} & \frac{7}{5}   \\
        0 & 0  & -\frac{130}{23} & -\frac{114}{23}   \\
        0 & 0 & -\frac{50}{23} & -\frac{35}{23}   \\
    \end{bmatrix}
\]

\[
    \begin{bmatrix}
        5 & 1 & 1 & -1   \\
        0 & \frac{23}{5} & \frac{33}{5} & \frac{7}{5}   \\
        0 & 0  & -\frac{130}{23} & -\frac{114}{23}   \\
        0 & 0 & 0 & -\frac{5}{13}   \\
    \end{bmatrix}
\]

\[
U =
\begin{bmatrix}
        5 & 1 & 1 & -1   \\
        0 & \frac{23}{5} & \frac{33}{5} & \frac{7}{5}   \\
        0 & 0  & -\frac{130}{23} & -\frac{114}{23}   \\
        0 & 0 & 0 & -\frac{5}{13}   \\
    \end{bmatrix}
\]


La matriz L antes del segundo intercambio de filas era
\[
\begin{bmatrix}
        1 & 0 & 0 & 0   \\
        \frac{1}{5} & 1 & 0 & 0   \\
        \frac{3}{5} & 0  & 1 & 0   \\
        -\frac{3}{5} & 0 & 0 & 1   \\
    \end{bmatrix}
\]

al cambiar las filas 2 y 3 resulta

\[
\begin{bmatrix}
        1 & 0 & 0 & 0   \\
        -\frac{3}{5} & 1 & 0 & 0   \\
        \frac{3}{5} & 0  & 1 & 0   \\
        \frac{1}{5} & 0 & 0 & 1  \\
    \end{bmatrix}
\]

con las nuevas operaciones elementales, antes del nuevo cambio resulta


\[
\begin{bmatrix}
        1 & 0 & 0 & 0   \\
        -\frac{3}{5} & 1 & 0 & 0   \\
        \frac{3}{5} &  \frac{2}{23}  & 1 & 0   \\
        \frac{1}{5} &  \frac{19}{23} & 0 & 1  \\
    \end{bmatrix}
\]

la matriz vuelve a cambiar la permutacion de las filas 3 y 4

\[
\begin{bmatrix}
        1 & 0 & 0 & 0   \\
        -\frac{3}{5} & 1 & 0 & 0   \\
        \frac{1}{5} &  \frac{19}{23} & 0 & 1  \\
        \frac{3}{5} &  \frac{2}{23}  & 1 & 0   \\
    \end{bmatrix}
\]

finalmente L = 

\[
L =
\begin{bmatrix}
        1 & 0 & 0 & 0   \\
        -\frac{3}{5} & 1 & 0 & 0   \\
        \frac{1}{5} &  \frac{19}{23} & 0 & 1  \\
        \frac{3}{5} &  \frac{2}{23}  & 1 & 0   \\
    \end{bmatrix}
\]

para la matriz P observamos

\[
\begin{bmatrix}
    1 & 0 & 0 & 0 \\
    0 & 1 & 0 & 0 \\
    0 & 0 & 1 & 0 \\
    0 & 0 & 0 & 1 \\
    \end{bmatrix}
\]
\[
\begin{bmatrix}
    0 & 1 & 0 & 0 \\
    1 & 0 & 0 & 0 \\
    0 & 0 & 1 & 0 \\
    0 & 0 & 0 & 1 \\
    \end{bmatrix}
\]
\[
\begin{bmatrix}
    0 & 1 & 0 & 0 \\
    0 & 0 & 0 & 1 \\
    0 & 0 & 1 & 0 \\
    1 & 0 & 0 & 0 \\
    \end{bmatrix}
\]
\[
\begin{bmatrix}
    0 & 1 & 0 & 0 \\
    0 & 0 & 0 & 1 \\
    1 & 0 & 0 & 0 \\
    0 & 0 & 1 & 0 \\
    \end{bmatrix}
\]

se comprueba que

\[
\begin{bmatrix}
    0 & 1 & 0 & 0 \\
    0 & 0 & 0 & 1 \\
    1 & 0 & 0 & 0 \\
    0 & 0 & 1 & 0 \\
    \end{bmatrix}
\begin{bmatrix}
    1 & 4 & 0 & -4 \\
    5 & 1 & 1 & -1 \\
    3 & 1 & -1 & -2 \\
    -3 & 4 & 6 & 2 \\
    \end{bmatrix}
=
    \begin{bmatrix}
        1 & 0 & 0 & 0 \\
        -\frac{3}{5} & 1 & 0 & 0 \\
        \frac{1}{5} & \frac{19}{23} & 1 & 0 \\
        \frac{3}{5} & \frac{2}{23} & \frac{5}{23} & 1 \\
        \end{bmatrix}
    \begin{bmatrix}
        5 & 1 & 1 & -1 \\
        0 & \frac{23}{5} & \frac{33}{5} & \frac{7}{5} \\
        0 & 0 & -\frac{130}{23} & -\frac{114}{2} \\
        0 & 0 & 0 & \frac{5}{13} \\
        \end{bmatrix}
\]







\subsection{SEL metodo de crammer}



\subsubsection{SEL solucion unica}

\subsubsection{1er Ejemplo}

\[
    \begin{bmatrix}
        x & -3y & = & 2  \\
        0 & +5y & = & 10 \\
        \end{bmatrix}
\]
la martiz del coeficiente del sistema es
\[
    A = 
    \begin{bmatrix}
        1 & -3 \\
        1 & 5  \\
        \end{bmatrix}
\]
la matriz de incogmitas
\[
    X = 
    \begin{bmatrix}
      x \\
      y  \\
        \end{bmatrix}
\]
la matriz de terminos independientes
\[
    B =
    \begin{bmatrix}
      2 \\
      10  \\
    \end{bmatrix}
\]
calculamos el determinante de A

\[
    |A| =
    \begin{bmatrix}
       | 1 - 3 |  \\
       | 1   5 | \\
    \end{bmatrix}
    = 5 + 3
    = 8 != 0
\]
para poder aplicar la regla de crammer

la primera incognita es x cuyos coeficientes son los de la primera columna de A
la matriz A es como A pero cambiando dicha columna por b

\[
    A_1 =
    \begin{bmatrix}
        2 - 3  \\
        10   5  \\
    \end{bmatrix}
\]
calculamos x
\[
    x =
    \begin{bmatrix}
        2 - 3  \\
        10   5  \\
    \end{bmatrix}
    / | A |
    = 40/8 = 5
\]

la segunda incognita es Y y sus coeficientes son los de la segunda columna de A
tenemos que calcular el determinante de la matriz

\[
    A_2 =
    \begin{bmatrix}
        1 2  \\
        1  10  \\
    \end{bmatrix}
\]
calculamnos y
\[
    Y =
    \begin{bmatrix}
       | 1 2  | \\
       | 1  10|   \\
    \end{bmatrix}
    / | A |
    = 8/8 = 1
\]
por lo tanto la solucion del sistema es 
x= 5
y= 1


\subsubsection{2er Ejemplo}

\[
    \begin{bmatrix}
        x & +y & = & 1  \\
        2x & +y & = & 0 \\
        \end{bmatrix}
\]

la matriz de coeficiente del sistema esa

\[
A =
\begin{bmatrix}
        1 & 1   \\
        2 & 1 \\
        \end{bmatrix}
\]
La matriz en terminos independiuentes es 
\[
B =
\begin{bmatrix}
        1    \\
        0  \\
        \end{bmatrix}
\]
calculamos el determinante de A
\[
| A | =
\begin{bmatrix}
      | 1 1 |      \\
      | 2 1 |    \\
        \end{bmatrix}
        = 1 -2 = -1 != 0
\]

aplicando la regla de crammer la matriz A esa

\[
A_1 =
\begin{bmatrix}
        1 1   \\
        0 1 \\
        \end{bmatrix}
\]

calculamos x

\[
X =
\begin{bmatrix}
    |1 1|       \\
    |0 1|     \\
\end{bmatrix}
/ | A |
= 1/ -1 = -1
\] 

la matriz a2 es
\[
    A_2 =
    \begin{bmatrix}
        1 1   \\
        2 0 \\
    \end{bmatrix}
\]
calculamos y
\[
   Y =
    \begin{bmatrix}
       | 1 1 |   \\
       | 2 0 | \\
    \end{bmatrix}
    / | A |
    = -2/-1 = 2
\]

por lo tanto la solucion del sistema esa
\[
    x = -1
    y = 2
\]

\end{document}